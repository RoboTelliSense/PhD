\documentclass[12pt]{article}
\usepackage[margin=1.0in]{geometry} %page layout
\usepackage[usenames,dvipsnames]{color} %color
\definecolor{light-gray}{gray}{0.95}
\definecolor{darkgreen}{rgb}{0,0.4,0}
\usepackage{graphicx, subfigure} %figures
\usepackage{url, hyperref} %cross-referencing
\usepackage{amsmath, amssymb} %math
\usepackage{listings} %source code
\lstset{breaklines=true,
breakindent=0pt,
prebreak=\mbox{\tiny$\searrow$},
postbreak=\mbox{{\color{blue}\tiny$\rightarrow$}},
numbers=left,
commentstyle=\color{darkgreen},
numberblanklines=false,
frame=single,
captionpos=b,
backgroundcolor=\color{light-gray}}
\usepackage[3D]{movie15} %for movies (needs hyperref)
\author{Salman Aslam\\Georgia Tech}
\title{Factor Analysis}
\date{}
\begin{document}
\maketitle
\rule[0pt]{\textwidth}{1pt}
\tableofcontents
\rule[0pt]{\textwidth}{1pt}

\begin{enumerate}
\item \underline{Given.}  Observation vector, $\mathbf{x}  \in R^d$.
\item \underline{Goal}
Model $\mathbf{x}$ as
\begin{equation}
\mathbf{x} = \mathbf{W}\mathbf{z}+ \mathbf{\mu} + \mathbf{\epsilon}
\end{equation}

\begin{table}[h]
\centering
\begin{tabular}{| l | c | c | l |}\hline
\textbf{Name} & \textbf{Notation} & \textbf{Distribution} & \textbf{Remarks}\\\hline
latent variable (factors) & $\mathbf{z} \in R^M$ & $ \mathcal{N}(\mathbf{0},\mathbf{I})$ & \parbox{2in}{iid, Gaussian variables with unit variance}\\\hline
- & $ \mathbf{\epsilon}$ & $ \mathcal{N}(\mathbf{0},\sigma^2\mathbf{I})$ &-\\\hline
- & $\mathbf{x|z} $ & $ \mathcal{N}(\mathbf{Wz+\mu},\sigma^2\mathbf{I})$ &-\\\hline							
observations & $\mathbf{x} \in R^D$ & $ \mathcal{N}(\mathbf{\mu},\mathbf{WW}^T + \sigma^2\mathbf{I})$ &$D>M$ \\\hline
\end{tabular}
\caption{Notation}
\end{table}

 $M<D$, the \emph{factors} are latent, .

$\mathbf{\epsilon}\sim \mathcal{N}(\mathbf{0},\mathbf{\Psi})$.

$\Rightarrow \mathsf{t}\sim \mathcal{N}(\mathbf{\mu},\mathbf{WW^T\Psi})$.


								\begin{figure}
								\centering
								\fbox{\includegraphics[width=0.5\textwidth]{figs/SP_covarianceSpectrum.pdf}}
								\caption{Covariance matrix spectrum.  If the training set size $N$ is larger than the dimensionality of the data $D$, it is possible to model the data with a full covariance matrix.} 
								\label{fig:covariance_matrix_spectrum}
								\end{figure}

The latent variables explain the correlations between observation variables while $\psi_i$ represents variability unique to a particular $t_i$.

\item \underline{Difference from PCA}.  

\item \underline{Motivation}.  Factor analysis is a a type of linear, latent-variable model.  The motivation is that with $q<d$, the latent variables offer a more parsimonious explanation of the dependencies between the observations.


\item \underline{Solution}.  Use Maximum Likelihood to estimate parameters.


\end{enumerate}


Refer to \cite{1999_JNL_PPCA_Tipping} for more details.
\bibliographystyle{ieee}
\bibliography{MyCitations}
\end{document}