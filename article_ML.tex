\documentclass[12pt]{article}
\usepackage[margin=1.0in]{geometry} %page layout
\usepackage[usenames,dvipsnames]{color} %color
\definecolor{light-gray}{gray}{0.95}
\definecolor{darkgreen}{rgb}{0,0.4,0}
\usepackage{graphicx, subfigure} %figures
\usepackage{url, hyperref} %cross-referencing
\usepackage{amsmath, amssymb} %math
\usepackage{listings} %source code
\lstset{breaklines=true,
breakindent=0pt,
prebreak=\mbox{\tiny$\searrow$},
postbreak=\mbox{{\color{blue}\tiny$\rightarrow$}},
numbers=left,
commentstyle=\color{darkgreen},
numberblanklines=false,
frame=single,
captionpos=b,
backgroundcolor=\color{light-gray}}
\usepackage[3D]{movie15} %for movies (needs hyperref)
	\newenvironment{changemargin}[2]
	{
	  	\begin{list}{}
		{
			\setlength{\topsep}{0pt}%
			\setlength{\leftmargin}{#1}%
			\setlength{\rightmargin}{#2}%
			\setlength{\listparindent}{\parindent}%
			\setlength{\itemindent}{\parindent}%
			\setlength{\parsep}{\parskip}%
		}
	  	\item[]
		}
		{\end{list}
	}
\author{Salman Aslam\\Georgia Tech}
\title{Model Finding\\Using\\Curve Fitting as an Example}
\date{}
\begin{document}
\maketitle
\rule[0pt]{\textwidth}{1pt}
\tableofcontents
\rule[0pt]{\textwidth}{1pt}

%=================================
\section{Given}
%=================================
A set of input and output points of a system.

Specifically, we have a training set with $N$ data points $\mathsf{x} = (x_1, x_2, \ldots, x_N)^T$ and $N$ corresponding labels, some possibly incorrect, $\mathsf{t} = (t_1, t_2, \ldots, t_N)^T$.


%=================================
\section{Goal}
%=================================
Find the mapping (model) between input and output.  Specifically for the curve fitting case, find a polynomial curve,

\begin{equation}
{\color{red}y(x,\mathbf{w})} = w_0 + w_1x + w_2 x^2 + \ldots + w_Mx^M = \sum\limits_{j=0}^M w_jx^j
\end{equation}

that best describes the training set.

%=================================
\section{Solution}
%=================================
\begin{enumerate}

%least squares%
{\color{blue}\item \underline{Least squares approach.}}
\begin{enumerate}
\item \underline{Model}.  Minimize the vertical distance error shown in Figure~\ref{fig:curve_fitting} and given by,
\begin{equation}
E(\mathbf{w}) = \sum\limits_{n=1}^N\big({\color{red}y(x_n,\mathbf{w})} - t_n\big)^2
\end{equation}
This error function is quadratic in $\mathbf{w}$.  Its derivatives wrt $\mathbf{w}$ are therefore linear and a unique minimum can therefore be found.
\item \underline{Goal}.  Estimate $\mathbf{w}$.
\item \underline{Solution}.  Construct $\mathbf{H}$ from $\mathsf{x}$.  Then,
\begin{equation}
\mathbf{w} = (\mathbf{H^TH})^{-1} \mathbf{H^T}\mathsf{t}
\end{equation}
\item \underline{Shortcomings}.  
\begin{itemize}
\item Determining $M$, the \emph{order} of the polynomial, i.e., \emph{model selection}.
\item This is a specific case of ML and therefore suffers from over-fitting.    
\end{itemize}
\end{enumerate}

%least squares with regularization%
{\color{blue}\item \underline{Least squares approach with regularization.}}
\begin{enumerate}
\item \underline{Model}.
\begin{equation}
E(\mathbf{w}) = \sum\limits_{n=1}^N\big({\color{red}y(x_n,\mathbf{w})} - t_n\big)^2 + \lambda{\|\mathbf{w}^2\|}
\end{equation}
\end{enumerate}

								\begin{figure}[t]
								\centering
								\subfigure[Deterministic approach: least squares.]{\includegraphics[width=0.4\textwidth]{figs/CV_curveFitting.png}}
								\subfigure[Probabilistic approach: maximum likelihood.]{\includegraphics[width=0.47\textwidth]{figs/CV_curveFitting2.png}}
								\caption{Curve fitting~\cite{2007_BOOK_PRML_Bishop}.}
								\label{fig:curve_fitting}
								\end{figure}


%max likelihood%
{\color{blue}\item \underline{Maximum Likelihood.}}  
\begin{enumerate}
\item \underline{Model}.  Consider each of the $N$ labels, $t_1, t_2, \ldots, t_N$ a separate gaussian random variable,  
\begin{equation}
p(t|x, \mathbf{w}, \beta) = \mathcal{N}(t|{\color{red}y(x,\mathbf{w})}, \beta^{-1})
\end{equation}
One of these variables is depicted in Figure~\ref{fig:curve_fitting}.  The joint distribution of these probabilistic labels, the \emph{total likelihood function} is given by,
\begin{equation}
\begin{array}{lllllllll}
p(\mathsf{t}|\mathsf{x}, \mathbf{w}, \beta) &= \prod\limits_{n=1}^N\mathcal{N}(t_n|{\color{red}y(x_n,\mathbf{w})}, \beta^{-1})\\
&=\mathcal{N}(t_1|{\color{red}y(x_1,\mathbf{w})}, \beta^{-1}).\mathcal{N}(t_2|{\color{red}y(x_2,\mathbf{w})}, \beta^{-1}). \ \ \ldots \ \ \mathcal{N}(t_n|y(x_n,\mathbf{w}), \beta^{-1})\\
&=\frac{1}{(\sqrt{2\pi})^N |\Sigma|^{1/2}}\exp\big\{-\frac{1}{2}({\color{red}y(\mathsf{x},\mathbf{w})}-t)^T\Sigma^{-1}({\color{red}y(\mathsf{x},\mathbf{w})}-t)\big\}
\end{array}
\end{equation}

\item \underline{Goal}.  Estimate $\mathbf{w}$ and $\beta$.
\item \underline{Solution}. 
\begin{align}
\ln p(\mathsf{t}|\mathsf{x}, \mathbf{w}, \beta) &= -\frac{\beta}{2} \sum\limits_{n=1}^N\big({\color{red}y(x_n,\mathbf{w})} - t_n\big)^2 + \frac{N}{2}\ln(\beta) -  \frac{N}{2}\ln(2\pi)\\\notag
\frac{1}{\beta} &= \frac{1}{N} \sum\limits_{n=1}^N\big({\color{red}y(x_n,\mathbf{w})} - t_n\big)^2
\end{align}
\item \underline{Relationship with notation in Kay~\cite{1993_BOOK_SSP_Kay}}.
So, ML under the Gaussian noise assumption is equivalent o minimizing least squares.


								\begin{figure}[t]
								\centering
								\includegraphics[width=1.0\textwidth]{figs/CV_ML_notation.pdf}
								\caption{Notation difference in Kay~\cite{1993_BOOK_SSP_Kay} and Bishop~\cite{2007_BOOK_PRML_Bishop}.}
								\label{fig:curve_fitting}
								\end{figure}

\end{enumerate}



%semi-Bayesian%
{\color{blue}\item \underline{Semi-Bayesian.}}  
\item \underline{Model}.
\begin{equation}
p(\mathsf{w}| \mathsf{x, t}, \alpha, \beta) = 
\end{equation}



%Bayesian%
{\color{blue}\item \underline{Bayesian.}} 


\end{enumerate}




\bibliographystyle{ieee}
\bibliography{MyCitations}
\end{document}