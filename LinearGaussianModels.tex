\documentclass[12pt]{article}
\usepackage[margin=1.0in]{geometry} %page layout
\usepackage[usenames,dvipsnames]{color} %color
\definecolor{light-gray}{gray}{0.95}
\definecolor{darkgreen}{rgb}{0,0.4,0}
\usepackage{graphicx, subfigure} %figures
\usepackage{url, hyperref} %cross-referencing
\usepackage{amsmath, amssymb} %math
\usepackage{listings} %source code
\lstset{breaklines=true,
breakindent=0pt,
prebreak=\mbox{\tiny$\searrow$},
postbreak=\mbox{{\color{blue}\tiny$\rightarrow$}},
numbers=left,
commentstyle=\color{darkgreen},
numberblanklines=false,
frame=single,
captionpos=b,
backgroundcolor=\color{light-gray}}
\usepackage[3D]{movie15} %for movies (needs hyperref)
\author{Salman Aslam\\Georgia Tech}
\title{Linear Gaussian Models}
\date{}
\begin{document}
\maketitle
\rule[0pt]{\textwidth}{1pt}
\tableofcontents
\rule[0pt]{\textwidth}{1pt}
%=========================
\section{Introduction}
%=========================

It has been shown by Roweis and Ghahramani~\cite{1999_JNL_Gaussian_roweis} that under the assumption of gaussian noise, factor analysis (FA), principal component analysis (PCA), mixtures of gaussian (MoG) clusters, vector quantization (VQ), independent components analysis (ICA), Kalman filters and hidden Markov models (HMMs) are instances of a single basic generative model, the linear gaussian model.  The linear gaussian model can be written as,

\begin{equation}
\begin{array}{llllllllllllll}
\mathbf{x}_{t+1} &=  \mathbf{A}\mathbf{x}_{t} +  \mathbf{w}_t   & & & \mathbf{w}_t \sim \mathcal{N}(0, \mathbf{Q})\\
\mathbf{y}_t 		 &=  \mathbf{C}\mathbf{x}_{t} +  \mathbf{v}_t    & & & \mathbf{v}_t \sim \mathcal{N}(0, \mathbf{R})
\end{array}
\label{LGM}
\end{equation}

where $\mathbf{A}$ is the state transition matrix, $\mathbf{C}$ is the observation matrix, and $\mathbf{w}_t$ and $\mathbf{v}_t$ are zero-mean gaussian random variables.  Under conditions where each data point $\mathbf{x}$ was generated independently and identically without any temporal ordering, i.e., $\mathbf{A}=\mathbf{0}$, we can write,

\begin{equation}
\begin{array}{llllllllllllll}
\mathbf{x} &= \mathbf{w} 											& & & \mathbf{w} &\sim \mathcal{N}(0, \mathbf{Q})\\
\mathbf{y} &=  \mathbf{C}\mathbf{x} +  \mathbf{v} 		& & & \mathbf{v} & \sim \mathcal{N}(0, \mathbf{R})
\end{array}
\label{LGM1}
\end{equation}



\begin{table}[t]
\centering
\begin{tabular}{| c | c | c | c | c |}\hline
 				 	&\textbf{A}	 	&	\textbf{C}  									& \textbf{Q} 	&  \textbf{R}                                                                		\\\hline
\textbf{PCA} 	&\textbf{0}		&	principal eigenvectors of $\boldsymbol\Sigma$	& \textbf{I}  	&  $\lim\limits_{\sigma^2 \rightarrow 0} \sigma^2\mathbf{I}$ 	\\\hline
\textbf{PPCA} &\textbf{0}		& 	scaled principal eigenvectors of $\boldsymbol\Sigma$	& \textbf{I}	&										      $\sigma^2 \mathbf{I}$	 \\\hline
\textbf{FA}   	&\textbf{0}		&													& \textbf{I} 	&  diagonal matrix 																\\\hline
\textbf{VQ}	 	&\textbf{0} 	&	cluster means								& \textbf{I}	& 	-																					\\\hline
\end{tabular}
\caption{Unifying PCA, PPCA, FA, and VQ using linear gaussian models~\cite{1999_JNL_Gaussian_roweis, 1999_JNL_PPCA_Tipping}.}
\label{table:LGM_unifying}
\end{table}

Moreover, for non-linear mappings, such as in MoG and VQ, a function $\mathbf{WTA[.]}$, \emph{winner-take-all} is introduced which returns a vector with unity in one position and all remaining zeros,

\begin{equation}
\begin{array}{llllllllllllll}
\mathbf{x} &= \mathbf{WTA}(\mathbf{w}) 						& & & \mathbf{w} &\sim \mathcal{N}(\mathbf{\boldsymbol\mu}, \mathbf{Q})\\
\mathbf{y} &=  \mathbf{C}\mathbf{x} +  \mathbf{v} 		& & & \mathbf{v} & \sim \mathcal{N}(0, \mathbf{R})
\end{array}
\label{LGM2}
\end{equation}

The values for \textbf{A}, \textbf{C}, \textbf{Q} and \textbf{R} under this elegant, unifying generative framework for various algorithms, including PCA and VQ, are given in Table~\ref{table:LGM_unifying}.  Besides being inherently satisfying, this framework allows the computation of data likelihoods in the case of PCA and VQ since they both do not define a proper density in the observation space~\cite{1999_JNL_Gaussian_roweis}.  We start with likelihoods under PCA, PPCA and VQ.  Our contribution is extending this work to compute likelihoods under RVQ.

For PCA, using Equation~\ref{LGM1} and Table~\ref{table:LGM_unifying}, we have the likelihood of observing $\mathbf{y}$ given by,

\begin{equation}
p(\mathbf{y}) \sim \mathcal{N}(\boldsymbol\mu, \mathbf{C}\mathbf{C}^T + \sigma^2 \mathbf{I})
\end{equation}

Bishop and Tipping~\cite{1999_JNL_PPCA_Tipping} show that this can be done in closed form using PPCA with the following solution,

\begin{equation}
\begin{array}{lllll}
\mathbf{\boldsymbol\mu}_{\textrm{ML}} &=\frac{1}{N}\sum\limits_{i=1}^N \mathbf{x}_i\\
\sigma^2_{\textrm{ML}} &= \frac{1}{D-q}\sum\limits_{i=q+1}^D \lambda_i\\
\mathbf{C}_{\textrm{ML}} &= \mathbf{U}_q(\mathbf{\Lambda}_q - \sigma^2\mathbf{I})^{1/2} \\
\end{array}
\label{Eqn:PPCA}
\end{equation}

where, $\boldsymbol\Sigma = \mathbf{U}\mathbf{\Lambda}\mathbf{V}^T$, $\mathbf{U}_q$ are the first $q$ eigenvectors in $\mathbf{U}$, $\mathbf{\Lambda}_q$ contains the corresponding eigenvalues, $\lambda_1, \lambda_2, \ldots \lambda_q$, $D$ is the dimensionality of the data and $N$ is the number of data points.  Note that in the above equation, we have omitted multiplication with an arbitrary rotation matrix since it can be taken to be $\mathbf{I}$ without loss of generality.  

In Equation~\ref{Eqn:PPCA}, $\sigma^2_{\textrm{ML}}$ is the average variance in the discarded dimensions and $\mathbf{C}_{\textrm{ML}}$ maps $\mathbf{x}$ onto its principal components to give output $\mathbf{y}$.

Moreover, in the linear gaussian model, Table~\ref{table:LGM_unifying} shows that in VQ, similar to PCA, the observation noise vanishes, $\lim\limits_{\sigma^2 \rightarrow 0} \sigma^2\mathbf{I}$.  The posterior therefore collapses to a single point, and all its mass is centered on the nearest centroid.  In this case, the likelihood of a data point $\mathbf{x}_i$ is proportional to its squared distance from the nearest centroid.


\clearpage
\newpage
\normalsize
\bibliographystyle{ieee}
\bibliography{MyCitations}
\end{document}
