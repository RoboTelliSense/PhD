%@@@@@@@@@@@@@@@@@@@@@@@@@@@@@@@@@@@@@@@@@@@@@@@@@@
\chapter{Conclusions}
\label{chap_conclusions}	
%@@@@@@@@@@@@@@@@@@@@@@@@@@@@@@@@@@@@@@@@@@@@@@@@@@
The results in the previous chapter show:

\begin{enumerate}
\item 
\end{enumerate}

%#################################
\section{Summary of work}
%#################################
The following is a summary of the work in this thesis:

\begin{enumerate}
\item Successful application of RVQ for 2 forms of video analysis: (a) human action recognition, and (b) visual tracking
\item For human action recognition, RVQ was used as a transform to represent an image as a $T$-tuple.  Every action sequence for every user was Each user's action One HMM per action sequence was trained on the RVQ $T$-tuple
\end{enumerate}
%#################################
\section{Contributions}
%#################################
This work has 2 contributions:

\begin{enumerate}
\item A demonstration that RVQ can be used effectively for visual tracking under a variety of appearance changes
\item Comparison of RVQ tracking with PCA and TSVQ tracking
\end{enumerate} 

%#################################
\section{Rationale}
%#################################
We have based our design on a well-known method for visual tracking~\cite{2008_JNL_subspaceTRK_Ross}.  The advantage of using an existing tracking framework is that it allows this approach is that since RVQ has never been used for visual tracking, the tracking framework is 
%#################################
\section{Next steps}
%#################################