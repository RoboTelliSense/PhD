\documentclass{article}
\usepackage{amsmath,epsfig}
\usepackage{subfigure}
\usepackage{color}
\usepackage[top=1in, bottom=1in, left=1in, right=1in]{geometry}
\definecolor{darkgreen}{rgb}{0,0.5,0}
\newcommand{\Ntrg}{\big[N_{t=1, m=1} + \lambda \big] + \big[N_{t=1, m=2} + \lambda \big] + \ldots + \big[N_{t=1, m=M} + \lambda \big]}
\newcommand{\jointcnt}{\sum\limits_{n_{trg}=1}^{N_{trg}}I(X_t=x_t, X_{t-1}=x_{t-1})}
\newcommand{\singlecnt}{\sum\limits_{n_{trg}=1}^{N_{trg}}I(X_{t-1}=x_{t-1})}
\newcommand{\singlep}{p(X_{t-1}=x_{t-1})}
\newcommand{\singlepone}{p(X_{t-1}=1)}
\newcommand{\singleptwo}{p(X_{t-1}=2)}
\newcommand{\singlepM}{p(X_{t-1}=M)}
\newcommand{\condp}{p(X_t=x_t | X_{t-1}=x_{t-1})}
\newcommand{\jointp}{p(X_t=x_t, X_{t-1}=x_{t-1})}
\newcommand{\KmeansOuterSum}{\sum\limits_{k=1}^K}
\newcommand{\KmeansInnerSum}{\sum\limits_{{i=1 \atop x_i \in \mathcal{K}_k}}^N}
\newcommand{\KmeansSum}{\KmeansOuterSum \KmeansInnerSum}
\newcommand{\RVQInnerSum}{\sum\limits_{{i=1 \atop g_i \mapsto m_{\tau, s}}}^N}
\newcommand{\RVQOuterSum}{\sum_{s=1}^S}
\newcommand{\RVQsum}{\KmeansOuterSum \sum\limits_{{i=1 \atop g_i \in \mathcal{K}_k}}^N}
\newcommand{\KmeansInner}{{(x_i - \mu_k)}^2}
\newcommand{\RVQinner}{            {(x_i  - \hat{\mu}^{(k)})}^2}
\newcommand{\RVQinneralternate}{{(g_i - m_\tau^{(k)})}^2}
\newcommand{\RVQinneralternatealternate}{{(g_i - m_{\tau, s})}^2}
\newcommand{\KmeansError}{\KmeansSum \KmeansInner}
\newcommand{\RVQerror}     {\KmeansSum \RVQinner}
\newcommand{\RVQerroralternate}{\RVQsum \RVQinneralternate}
\newcommand{\RVQunit}{x_i -\bigg(\sum_{t=1}^Tm^{(k)}_t\bigg)}
\newcommand{\RVQequivalentCodevector}{\sum_{t=1 }^Tm^{(k)}_t}
\newcommand{\RVQequivalentCodevectorBroken}{\sum_{t=1 \atop t \neq \tau}^Tm^{(k)}_t+ m^{(k)}_\tau}
\newcommand{\RVQmultipleKmeans}{x_i -\bigg(\RVQequivalentCodevectorBroken\bigg)}
\newcommand{\RVQmultipleKmeansone}{x_i -\sum_{t=2}^Tm^{(k)}_t+ m^{(k)}_1\bigg)}
\newcommand{\RVQmultipleKmeansonealternate}{\bigg(x_i -\sum_{t=1 \atop t \neq \tau}^Tm^{(k)}_t\bigg) - m^{(k)}_\tau}
\newcommand{\RVQmultipleKmeanstwo}{x_i -\bigg(\sum_{t=1 \atop t \neq 2}^Tm^{(k)}_t+ m^{(k)}_2\bigg)}
\newcommand{\RVQmultipleKmeansT}{x_i -\bigg(\sum_{t=1}^{T-1}m^{(k)}_t+ m^{(k)}_2\bigg)}
\newcommand{\EucMatrix}
{
\left[
\begin{array}{lll}
r_{11} & r_{12} & t_x \\ 
r_{21} & r_{22} & t_y \\ 
0 & 0 & 1 \\ 
\end{array}
\right]
}	

\newcommand{\SimMatrix}
{
\left[
\begin{array}{lll}
sr_{11} & sr_{12} & t_x \\ 
sr_{21} & sr_{22} & t_y \\
0 & 0 & 1 \\ 
\end{array}
\right]
}

\newcommand{\AffMatrix}
{
\left[
\begin{array}{lll}
a &b & t_x \\ 
c & d & t_y \\
0 & 0 & 1 \\
\end{array}
\right]
}

\newcommand{\ProjMatrix}
{
\left[
\begin{array}{lll}
h_{11} & h_{12} & h_{13} \\ 
h_{21} & h_{22} & h_{23} \\ 
h_{31} & h_{32} & h_{33} \\ 
\end{array}
\right]
}

\newcommand{\RotMatrixTheta}
{
\left[
\begin{array}{rr}
\cos(\theta) & -\sin(\theta) \\ 
\sin(\theta) & \cos(\theta) \\ 
\end{array}
\right]
}

\newcommand{\RotMatrixPhi}
{
\left[
\begin{array}{rr}
\cos(\phi) & -\sin(\phi) \\ 
\sin(\phi) & \cos(\phi) \\ 
\end{array}
\right]
}

\newcommand{\RotMatrixminusPhi}
{
\left[
\begin{array}{rr}
\cos(-\phi) & -\sin(-\phi) \\ 
\sin(-\phi) & \cos(-\phi) \\ 
\end{array}
\right]
}


\newcommand{\EigenvalueMatrix}
{
\left[
\begin{array}{cc}
\lambda_1 & 0\\
0 & \lambda_2
\end{array}
\right]
}

\newcommand{\bigMatrix}
{
s \left[
\begin{array}{cc}
 (r)(a) + b &  (r)(d) - c \\
 (r)(c) - d &  (r)(b) + a
\end{array}
\right]
}


\newcommand{\bigMatrixTwo}
{
\left[
\begin{array}{cc}
(\lambda_2) p + (\lambda_1) q & (\lambda_2) s  - (\lambda_1) r \\
(\lambda_2) r  - (\lambda_1) s & (\lambda_2) q + (\lambda_1) p
\end{array}
\right]
}
\newcommand{\dr}{(\mathbf{x}_i-\boldsymbol\mu_k)^T(\mathbf{x}_i-\boldsymbol\mu_k) + \lambda({Q_{\textrm{max}}-Q_i})}

\title{MCMC}
\begin{document}
\date{}
\maketitle

%=============================================
\section{Gibbs Sampler}
%=============================================
\vspace{0.5in}
\begin{enumerate}
\item \textbf{\underline{Given}}:
\begin{enumerate}
\item \underline{Random variables}.  $T$ random variables, $X_1, X_2, \ldots, X_T$.
\item \underline{Conditional distributions}.  Complete conditional distributions, $p(X_s|X_t), s \neq t, t=1, 2, \ldots, T$.
\item \underline{Possible to sample}.  It is possible to sample from the conditional distributions.
\item \underline{Joint distribution}.  Under mild conditions (Besag 1974), these complete conditional distributions uniquely determine the full joint distribution $p(X_1, X_2, \ldots, X_T)$, and hence all marginal distributions $p(X_t)$.
\item \underline{Initial samples}. Arbitrary initial samples, $X_{1(1)}, X_{2(1)}, X_{3(1)}, \ldots X_{T(1)}$
\end{enumerate}

\vspace{0.5in}
\item \textbf{\underline{Compute}}:
Given the above, the Gibbs sampler can generate a \textbf{single} sample from the joint distribution using the steps below,

\begin{enumerate}
\item \underline{Given initial samples, draw one sample from each conditional distribution.} Draw each of the following samples:
\begin{itemize}
\item $X_{1(2)}$ from $p(X_1  |  X_{2(1)}, X_{3(1)}, \ldots, , X_{T(1)})$
\item $X_{2(2)}$ from $p(X_2  |  X_{1(2)}, X_{3(1)}, \ldots, , X_{T(1)})$
\item $\vdots$
\item $X_{T(2)}$ from $p(X_T  |  X_{1(2)}, X_{3(2)}, \ldots, , X_{{T-1}(2)})$
\end{itemize}
\item \underline{Repeat above step for $L$ iterations.}  After $L$ such iterations, we will have a $T$-tuple, $\big(X_{1(L)}, X_{2(L)}, X_{3(L)}, \ldots, , X_{T(L)}\big)$.  Geman and Geman (1984) showed that under mild conditions, this $T$-tuple converges in distribution to a random observation from $p(X_1, X_2, \ldots, X_T)$ as $L \rightarrow \infty$.
\end{enumerate}
\end{enumerate}

\vspace{0.2in}
Repeat this process $G$ times to get $G$ samples from the joint density.  Refer to~\cite{1992_JNL_MCMC_Carlin} for more details.

\vspace{0.5in}
%We have a \emph{proposal density},
%
%\begin{equation}
%\end{equation}
%
%We start with a sample at $t=0, x_0$
%
%MH is a 2 step process:
%\begin{enumerate}
%\item Proposal value is generated
%\item Proposal value is accepted/rejected
%\begin{equation}
%a = \frac{p(x')}{p(x_t)}\frac{Q(x_t;x')}{Q(x';x_t)}
%\end{equation}
%\item if $a \geq 1, x_{t+1}=x'$
%\end{enumerate}


\bibliographystyle{ieee}
\bibliography{MyCitations}
\end{document}