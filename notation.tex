\documentclass{article}
\usepackage{longtable}
\usepackage{color}

\definecolor{darkgreen}{rgb}{0,0.5,0}
\newcommand{\Ntrg}{\big[N_{t=1, m=1} + \lambda \big] + \big[N_{t=1, m=2} + \lambda \big] + \ldots + \big[N_{t=1, m=M} + \lambda \big]}
\newcommand{\jointcnt}{\sum\limits_{n_{trg}=1}^{N_{trg}}I(X_t=x_t, X_{t-1}=x_{t-1})}
\newcommand{\singlecnt}{\sum\limits_{n_{trg}=1}^{N_{trg}}I(X_{t-1}=x_{t-1})}
\newcommand{\singlep}{p(X_{t-1}=x_{t-1})}
\newcommand{\singlepone}{p(X_{t-1}=1)}
\newcommand{\singleptwo}{p(X_{t-1}=2)}
\newcommand{\singlepM}{p(X_{t-1}=M)}
\newcommand{\condp}{p(X_t=x_t | X_{t-1}=x_{t-1})}
\newcommand{\jointp}{p(X_t=x_t, X_{t-1}=x_{t-1})}
\newcommand{\KmeansOuterSum}{\sum\limits_{k=1}^K}
\newcommand{\KmeansInnerSum}{\sum\limits_{{i=1 \atop x_i \in \mathcal{K}_k}}^N}
\newcommand{\KmeansSum}{\KmeansOuterSum \KmeansInnerSum}
\newcommand{\RVQInnerSum}{\sum\limits_{{i=1 \atop g_i \mapsto m_{\tau, s}}}^N}
\newcommand{\RVQOuterSum}{\sum_{s=1}^S}
\newcommand{\RVQsum}{\KmeansOuterSum \sum\limits_{{i=1 \atop g_i \in \mathcal{K}_k}}^N}
\newcommand{\KmeansInner}{{(x_i - \mu_k)}^2}
\newcommand{\RVQinner}{            {(x_i  - \hat{\mu}^{(k)})}^2}
\newcommand{\RVQinneralternate}{{(g_i - m_\tau^{(k)})}^2}
\newcommand{\RVQinneralternatealternate}{{(g_i - m_{\tau, s})}^2}
\newcommand{\KmeansError}{\KmeansSum \KmeansInner}
\newcommand{\RVQerror}     {\KmeansSum \RVQinner}
\newcommand{\RVQerroralternate}{\RVQsum \RVQinneralternate}
\newcommand{\RVQunit}{x_i -\bigg(\sum_{t=1}^Tm^{(k)}_t\bigg)}
\newcommand{\RVQequivalentCodevector}{\sum_{t=1 }^Tm^{(k)}_t}
\newcommand{\RVQequivalentCodevectorBroken}{\sum_{t=1 \atop t \neq \tau}^Tm^{(k)}_t+ m^{(k)}_\tau}
\newcommand{\RVQmultipleKmeans}{x_i -\bigg(\RVQequivalentCodevectorBroken\bigg)}
\newcommand{\RVQmultipleKmeansone}{x_i -\sum_{t=2}^Tm^{(k)}_t+ m^{(k)}_1\bigg)}
\newcommand{\RVQmultipleKmeansonealternate}{\bigg(x_i -\sum_{t=1 \atop t \neq \tau}^Tm^{(k)}_t\bigg) - m^{(k)}_\tau}
\newcommand{\RVQmultipleKmeanstwo}{x_i -\bigg(\sum_{t=1 \atop t \neq 2}^Tm^{(k)}_t+ m^{(k)}_2\bigg)}
\newcommand{\RVQmultipleKmeansT}{x_i -\bigg(\sum_{t=1}^{T-1}m^{(k)}_t+ m^{(k)}_2\bigg)}
\newcommand{\EucMatrix}
{
\left[
\begin{array}{lll}
r_{11} & r_{12} & t_x \\ 
r_{21} & r_{22} & t_y \\ 
0 & 0 & 1 \\ 
\end{array}
\right]
}	

\newcommand{\SimMatrix}
{
\left[
\begin{array}{lll}
sr_{11} & sr_{12} & t_x \\ 
sr_{21} & sr_{22} & t_y \\
0 & 0 & 1 \\ 
\end{array}
\right]
}

\newcommand{\AffMatrix}
{
\left[
\begin{array}{lll}
a &b & t_x \\ 
c & d & t_y \\
0 & 0 & 1 \\
\end{array}
\right]
}

\newcommand{\ProjMatrix}
{
\left[
\begin{array}{lll}
h_{11} & h_{12} & h_{13} \\ 
h_{21} & h_{22} & h_{23} \\ 
h_{31} & h_{32} & h_{33} \\ 
\end{array}
\right]
}

\newcommand{\RotMatrixTheta}
{
\left[
\begin{array}{rr}
\cos(\theta) & -\sin(\theta) \\ 
\sin(\theta) & \cos(\theta) \\ 
\end{array}
\right]
}

\newcommand{\RotMatrixPhi}
{
\left[
\begin{array}{rr}
\cos(\phi) & -\sin(\phi) \\ 
\sin(\phi) & \cos(\phi) \\ 
\end{array}
\right]
}

\newcommand{\RotMatrixminusPhi}
{
\left[
\begin{array}{rr}
\cos(-\phi) & -\sin(-\phi) \\ 
\sin(-\phi) & \cos(-\phi) \\ 
\end{array}
\right]
}


\newcommand{\EigenvalueMatrix}
{
\left[
\begin{array}{cc}
\lambda_1 & 0\\
0 & \lambda_2
\end{array}
\right]
}

\newcommand{\bigMatrix}
{
s \left[
\begin{array}{cc}
 (r)(a) + b &  (r)(d) - c \\
 (r)(c) - d &  (r)(b) + a
\end{array}
\right]
}


\newcommand{\bigMatrixTwo}
{
\left[
\begin{array}{cc}
(\lambda_2) p + (\lambda_1) q & (\lambda_2) s  - (\lambda_1) r \\
(\lambda_2) r  - (\lambda_1) s & (\lambda_2) q + (\lambda_1) p
\end{array}
\right]
}
\newcommand{\dr}{(\mathbf{x}_i-\boldsymbol\mu_k)^T(\mathbf{x}_i-\boldsymbol\mu_k) + \lambda({Q_{\textrm{max}}-Q_i})}


\begin{document}

\begin{longtable}{llllp{3in}}
\textbf{\color{red} \underline{general}}&&&&\\
&$C$   						&=&& number of classes, $c=1, 2, \ldots C$						\\
&$\mathcal{C}_c$			&=&& $c$-th class 												\\
&$N$   					&=&& number of data points, $n =1, 2, \ldots N	$				\\
&$N_{trg}$   				&=&& number of training examples, $a =1, 2, \ldots N_{trg}$	\\
&$N_{tst}$  				&=&& number of test examples, $b= 1, 2, \ldots N_{tst}$		\\
&$H$ 						&=&& entropy														\\ 
&$I(.)$						&=&& indicator function											\\\\


\textbf{\color{red} \underline{images}}&&&&																				\\
&\textbf{\underline{input image}}&&&																						\\	
&$\mathbf{I}_f$    		&=&& at frame $f$																				\\
&$\mathbf{I}_w$    		&=&& width																						\\
&$\mathbf{I}_h$    		&=&& height																						\\
&$\mathbf{I}_N$    		&=&& number of pixels, $\mathbf{I}_N=\mathbf{I}_w*\mathbf{I}_h$							\\
&$\mathbf{I}_c$    		&=&& number of channels																		\\ \\

&\textbf{\underline{inner input image}}&&&																					\\
&$\mathbf{I}_{if}$    		&=&& at frame $f$																				\\
&$\mathbf{I}_{il}$   		&=&& left-most pixel index, $\mathbf{I}_{il} = \frac{s_w-1}{2}$ (for C-style indexing), $\mathbf{I}_{il} = \frac{s_w-1}{2}+1$ (for Matlab-style indexing)										\\
&$\mathbf{I}_{ir}$   		&=&& right-most pixel index, $\mathbf{I}_{ir} = \mathbf{I}_w - \frac{s_w-1}{2} - 1$ (for C-style indexing), $\mathbf{I}_{ir} = \mathbf{I}_w - \frac{s_w-1}{2}$ (for Matlab-style indexing)	\\
&$\mathbf{I}_{it}$   		&=&& top-most pixel index, $\mathbf{I}_{it} = \frac{s_h-1}{2}$ (for C-style indexing), $\mathbf{I}_{it} = \frac{s_h-1}{2}+1$ (for Matlab-style indexing)										\\
&$\mathbf{I}_{ib}$   		&=&& bottom-most pixel index, $\mathbf{I}_{ib} = \mathbf{I}_h - \frac{s_h-1}{2} - 1$ (for C-style indexing), $\mathbf{I}_{ib} = \mathbf{I}_h - \frac{s_h-1}{2}$ (for Matlab-style indexing)\\
&$\mathbf{I}_{iw}$   		&=&& width, $\mathbf{I}_{iw} = \mathbf{I}_{ir} - \mathbf{I}_{il} + 1$ (for C-style and Matlab-style indexing) 																					\\
&$\mathbf{I}_{ih}$   		&=&& height, $\mathbf{I}_{ih} = \mathbf{I}_{ib} - \mathbf{I}_{it} + 1$ (for C-style and Matlab-style indexing) 																					\\
&$\mathbf{I}_{iN}$   		&=&& number of pixels, $\mathbf{I}_{iN}=\mathbf{I}_{iw}*\mathbf{I}_{ih}$					\\ \\

&\textbf{\underline{pixels}}&&&																								\\ \\




\textbf{\color{red} \underline{RVQ}}&&&&\\
&SoC 						&=&& Sum-of-Codevectors descriptor 																				\\
&$e$						&=&& mean squared error between inputs and reconstructed inputs												\\\\

&\textbf{\underline{dimensions}}&&&																											\\
&$T$    					&=&& number of stages in RVQ $\sigma$-tree, $t=1, 2, \ldots T$ 													\\
&$M$    					&=&& number of templates per stage in RVQ $\sigma$-tree															\\\\

&\textbf{\underline{clusters}}&&&																												\\
&$\mathcal{K}_k$			&=&& $k$-th cluster, $k=1, 2, \ldots K$																				\\
&$\mathcal{H}_{t, m}$	&=&& cluster of points that map to $\phi_{t, m}$ stage-codevector													\\\\

&$K$    					&=&& number of equivalent codevectors (number of clusters),  														\\
&$N_{\mathcal{K}_k}$ 	&=&& number of input data points that map to the $k$-th cluster													\\\\
&$N_{\mathcal{H}_{k, t, m}}$	
							&=&& number of data points from cluster $\mathcal{K}_k$ that also lie in cluster $\mathcal{H}_{t, m}$			\\\\

&\textbf{\underline{codevectors}}&&&																											\\
&$\hat{\mu}_k$			&=&& equivalent code-vector $k$-th mean, uses $k$-th SoC descriptor for construction								\\
&$\phi_{k(t)}$				&=&& stage code-vector, $t$-th additive component of $\hat{\mu}_k$												\\
&$\phi_{t, m}$			&=&& stage-codevector at $t$-th stage and $m$-th template  $\hat{\mu}_j=\sum\limits_{s=1}^{N_{stg}}\phi_s^{(j)} = \sum\limits_{s=1 \atop s \neq \xi}^{N_{stg}}\phi_s^{(j)} + \phi_\xi^{(j)}, j={1, 2, \dots, N_K}$														\\
&$e$						&=&&$\KmeansError = {(x_1 - \hat{x}_1)}^2 + {(x_2 - \hat{x}_2)}^2 + \ldots + {(x_N - \hat{x}_N)}^2$ 																										\\
&$\Phi$					&=&& codebooks																										\\
&$\Phi_s$					&=&& scaled codebooks so they can be viewed as images															\\\\



&\textbf{\underline{intermediate images}}&&&																									\\
&$\mathbf{I}_{snr}$			&=&& a 2D matrix (image) containing reconsruction SNR values of every pixel in inner image					\\
&$\mathbf{I}_{stg}$			&=&& a 2D matrix (image) containing reconsruction stages of every pixel in inner image						\\
&$\mathbf{I}_{lglik}$			&=&& a 2D matrix (image) containing log likelihoods of every pixel in inner image								\\\\



&\textbf{\underline{snippets}}&&&																												\\
&$s_{w}$   				&=&& width in pixels (always taken as odd)																			\\
&$s_{h}$  					&=&& height in pixels (always taken as odd)																			\\
&$s_{hc}$  				&=&& height in bytes of planar (non-interleaved) channels, $s_{hc} =s_h * s_c$									\\
&$s_{c}$  					&=&& number of channels																							\\
&$s_{b}$  					&=&& bytes, $s_{b} = s_{w} * s_{h} * s_{c}$																		\\ \\

&$N_{s_x}$   				&=&& training snippets, number per class in x direction 																\\
&$N_{s_y}$  	 			&=&& training snippets, number per class in y direction																\\
&$N_{s_x} * N_{s_y}$    	&=&& training snippets, total number per class, $N_{trg} = N_{s_x} * N_{s_y} * C$									\\
&$s_{dx}$   				&=&& training snippets, spacing in x direction																		\\
&$s_{dy}$  				&=&& training snippets, spacing in y direction																		\\ \\

&$s_X$  					&=&& training snippets, set of x-coordinates of snippet centers in input image										\\ 
&$s_Y$  					&=&& training snippets, set of y-coordinates of snippet centers in input image										\\
&$s_{iX}$  				&=&& training snippets, set of x-coordinates of snippet centers in input inner image								\\ 
&$s_{iY}$  				&=&& training snippets, set of y-coordinates of snippet centers in input inner image								\\ \\


&snippet extraction details file   				&=&& a comma separated file containing snippet extraction details, each row has the following 5 values in comma separated format: class\_label, image\_name, $I_w, I_h$, snippet\_center\_x, snippet\_center\_y, image name is the name of the image from which to extract the snippet, notice that $s_w$ and $s_h$ are not specified here\\ \\


&\textbf{\underline{SoC descriptors}}&&&																															\\
&$\mathbf{a}_a$  		&=&&  training SoC descriptor, a $T$-tuple 																																	\\
&$\mathbf{b}_b$ 			&=&&  test SoC descriptor, could be truncated if monotonic PSNR increase is a condition in reconstruction, in this case $M+1$ is used as filler to make it a $T$-tuple	\\
&$\mathbf{A}$  			&=&&  $N_{trg}$x$T$ matrix of training SoC descriptors, one observation, i.e. SOC descriptor $\mathbf{a}_a$ per row 													\\
&$\mathbf{B}$  			&=&&  $N_{tst}$x$T$ matrix of testing SoC descriptors, one observation, i.e. SOC descriptor $\mathbf{b}_b$ per row 													\\ \\

&\textbf{\underline{thresholds}}&&&																															\\
&$\gamma_{SNRfrac}$  		&=&&  reconstruction SNR fraction threshold 																																	\\
&$\gamma_{SNR}$  		&=&&  reconstruction SNR threshold 																																	\\
&$\gamma_{stg}$  		&=&&  reconstruction stages threshold 																																	\\
&$\gamma_{lglik}$  		&=&&  log likelihood threshold 																																	\\ \\

&\textbf{\underline{stage random variables}}&&&\\																															\\
&$X_t$ 					&=&& for a given column in $\mathbf{A}$, use its rows to create random variable $X_t$ for stage $t$ during training phase, $X_t = \{1,2 \ldots M \}$ 					\\
&$Y_t$ 					&=&& for a given column in $\mathbf{B}$, use its rows to create random variable $Y_t$ for stage $t$ during testing phase, $Y_t = \{1,2 \ldots M+1 \}$ 				\\
&$\mathbf{X}$  			&=&& random process, $\mathbf{X} = (X_1, X_2, \ldots, X_T)$																												\\
&$\mathbf{Y}$  			&=&& random process, $\mathbf{Y} = (Y_1, Y_2, \ldots, Y_T)$																												\\ \\
&$p(\mathbf{X}|\mathcal{C}_c)$ 
							&=&& class conditional density created from $\mathbf{A}$ 																													\\ 
&$p(X_1=1)$  				&=&& $\frac{N_{t=1, m=1}}{N_{trg}}$\\
&$p(\mathbf{Y})$ 		&=&& data probability																																							\\ \\ \\ \\


\textbf{\color{red} \underline{tracking}}&&&&\\
&\textbf{\underline{frames}}&&&	\\
&fn 						&=&& abbreviation for file name\\
&cfn 						&=&& abbreviation for complete file name, i.e. directory and filename\\\\

&$f$ 						&=&&  index of frames that need to be tracked, $f_I, \ldots, f_F$\\
&str\_f 						&=&&  a string that has enough zeros prefixed to $f$ to so that it is 5 characters long\\
&$f_{idx}$ 				&=&&  index for total frames processed, training and testing\\ 
&$f_{tst}$ 					&=&&  index for testing frames\\ \\

&\textbf{\underline{frames for target $t$}}&&&	\\
&$f_{tI}$ 					&=&& initial frame for target $t$\\
&$f_{tF}$ 					&=&& final frame for target $t$\\
&$f_{tN}$ 					&=&&  total number of frames in which target $t$ is tracked, $f_{tN}=f_{tF} - f_{tI} + 1$\\
&$f_t$ 						&=&&  index for target t, it is 1 when target $t$ enters frame, $f_t = 1, 2, ... f_{tN}$ \\ \\



&\textbf{\underline{bounding boxes}}\\	
&$R_{f,r}$ 				&=&& $(x, y, w, h)$, a 4-tuple describing the $r$-th BB in the $f$-th frame, $x, y$ are the coordinates of the top left corner\\
&$G_{f,g}$ 				&=&& $(x, y, w, h)$, a 4-tuple describing the $g$-th target's BB in the $f$-th frame, $x, y$ are the coordinates of the top left corner\\ \\
&$W_{f, g}$ 				&=&& $(x, y, w, h)$, a 4-tuple describing the $g$-th target's search window in the $f$-th frame, $x, y$ are the coordinates of the top left corner\\

&GT(g) 					&=&&  a matrix containing the $g$-th target's ground truth data, each row has the following $[f \  x \ y \ w \ h]$, $x, y$ are the coordinates of the top left corner\\
&BB 						&:&&  abbreviation for bounding box rectangle\\ \\


&$\mathcal{B}_{f,r}$ 		&=&& set of pixels that lie in $R_{f,r}$\\
&$\mathcal{G}_{f,g}$ 	&=&& set of pixels that lie in $G_{f,g}$\\ \\

&\textbf{\underline{ROC curves}}&&&	\\
&$\theta_{TP}(g)$ 		&=&& true positive for $g$-th target\\
&$\theta_{FN}(g)$ 		&=&& false negative for $g$-th target\\ 
&$\theta_{HR}(g)$ 		&=&& hit rate for $g$-th target\\ \\

&$\theta_{TN}(g)$ 		&=&& true negative for $g$-th target\\
&$\theta_{FP}(g)$ 		&=&& false positive for $g$-th target\\
&$\theta_{FAR}(g)$ 		&=&& false alarm (positive) rate for $g$-th target\\

\caption{Notation}
\label{tab:notation}
\end{longtable}


\end{document}