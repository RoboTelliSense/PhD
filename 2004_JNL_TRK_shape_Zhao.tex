%=================================
\subsection{2004: crowded (Zhao)}
%=================================
\begin{frame}
\frametitle{Prior work: crowded (Zhao)}
\framesubtitle{1. overview}
\logoCSIPCPL\mypagenum
\myFootnoteCitation{2004_JNL_TRK_shape_Zhao}{PAMI}
	%\begin{figure}
	%	\includegraphics[width=1.0\textwidth]{tables/TrackingPapers_SubspaceTracking_2006_PCA_Hog.pdf}
	%\end{figure}
\end{frame}




\begin{frame}
\frametitle{Prior work: crowded (Zhao)}
\framesubtitle{2. summary}
\logoCSIPCPL\mypagenum
\myFootnoteCitation{2004_JNL_TRK_shape_Zhao}{PAMI}
	\begin{itemize}
		\item prior locomotion model to assist posture estimation
	\end{itemize}
\end{frame}



\begin{frame}
\frametitle{Prior work: crowded (Zhao)}
\framesubtitle{3. comparison}
\logoCSIPCPL\mypagenum	
\myFootnoteCitation{2004_JNL_TRK_shape_Zhao}{PAMI}	
	\begin{itemize}
		\item difficulties with blob-based analysis (see pg \pageref{fig:1})
			\begin{enumerate}
				\item a single blob may contain multiple humans due to their physical proximity or due to camera viewing angle
				\item a single object may be fragmented into several blobs due to low color contrast 
				\item blobs may contain pixels corresponding to shadows or reflections
			\end{enumerate}
				\item blobs may go through frequent structural changes (split and merge) due to above problems
				\item this causes combinatorial search for temporal correspondence
				\item even if correspondence is established, what each trajectory corresponds to (e.g. object, part of an object, a few objects together) is still unknown
	\end{itemize}
\end{frame}




\begin{frame}
\frametitle{Prior work: crowded (Zhao)}
\framesubtitle{3. comparison (cont.)}
\logoCSIPCPL\mypagenum	
\myFootnoteCitation{2004_JNL_TRK_shape_Zhao}{PAMI}
	\begin{itemize}
		\item advantages of shape based approach used
			\begin{enumerate}
				\item detecting individual objects is a goal
				\item real entities do not undergo structural changes such as split and merge
				\item constraints on (e.g. shape, size, motion) can assist segmentation and tracking
				\item less sensitive to noise and parameters of low-level processing
				\item a camera model provides additional constraints
			\end{enumerate}
	\end{itemize}
\end{frame}


\begin{frame}
\frametitle{Prior work: crowded (Zhao)}
\framesubtitle{4. contributions}
\logoCSIPCPL\mypagenum
\myFootnoteCitation{2004_JNL_TRK_shape_Zhao}{PAMI}
\end{frame}


\begin{frame}
\frametitle{Prior work: crowded (Zhao)}
\framesubtitle{5. methodology: overview}
\logoCSIPCPL\mypagenum
\myFootnoteCitation{2004_JNL_TRK_shape_Zhao}{PAMI}
	\begin{enumerate}
		\item change detection
		\item human hypotheses are computed by boundary and shape analysis using human shape and camera model
		\item each hypothesis is tracked in 3D in the subsequent frames with a Kalman filter using the object's appearance constrained by its shape
		\item 2D positions are mapped onto the 3D ground plane and trajectories are formed and filtered in 3D
	\end{enumerate}
\end{frame}



\begin{frame}
\frametitle{Prior work: crowded (Zhao)}
\framesubtitle{5. methodology: detailed}
\logoCSIPCPL\mypagenum
\myFootnoteCitation{2004_JNL_TRK_shape_Zhao}{PAMI}
	\begin{enumerate}\setcounter{enumi}{0}
		\item change detection
			\begin{itemize}
				\item single gaussian like Pfinder
			\end{itemize}
		\item camera model
			\begin{itemize}
				\item linear calibration method (\mycite{2002_BOOK_CV_Forsyth}, \mycite{1999_CNF_ModelFromIMG_Liebowitz})
				\item self calibration from walking human (\mycite{2006_JNL_Camera_Lv})
			\end{itemize}
	\end{enumerate}
\end{frame}




\begin{frame}
\frametitle{Prior work: crowded (Zhao)}
\framesubtitle{figures}
\mypagenum
\myFootnoteCitation{2004_JNL_TRK_shape_Zhao}{PAMI}
	\begin{figure}
		\includegraphics[width=1.0\textwidth]{figs/TrackingPapers_shape_2004_Zhao_fig1.jpg}
		\label{fig:1}
	\end{figure}
\end{frame}



\begin{frame}
\frametitle{Prior work: crowded (Zhao)}
\framesubtitle{figures (cont.)}
\mypagenum
\myFootnoteCitation{2004_JNL_TRK_shape_Zhao}{PAMI}
	\begin{figure}
		\includegraphics[width=1.0\textwidth]{figs/TrackingPapers_shape_2004_Zhao_fig6.jpg}
	\end{figure}
\end{frame}